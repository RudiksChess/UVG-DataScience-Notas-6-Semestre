\input{Configuraciones/paquetes}

%--------------------------

\begin{document}
 \thispagestyle{empty} 
    \begin{tabular}{p{15.5cm}}
    \begin{tabbing}
    \textbf{Universidad del Valle de Guatemala} \\
    Departamento de Ciencias de la Computación\\\\
   \textbf{Estudiantes:} Augusto Alonso, Angel Cuellar, Rudik Roberto Rompich\\
    \end{tabbing}
    \begin{center}
        CC3066 - Data Science I - Catedrático: Luis Furlan\\
        \today
    \end{center}\\
    \hline
    \\
    \end{tabular} 
    \vspace*{0.3cm} 
    \begin{center} 
    {\Large \bf  Proyecto 2 - Análisis Exploratorio 
} 
        \vspace{2mm}
    \end{center}
    \vspace{0.4cm}
%--------------------------

\textbf{Instrucciones:} en clase vimos un modelo simple para resolver regresiones lineales mediante redes neuronales.  Utilizado el código desarrollado (o si lo desea uno propio), responda a las siguientes preguntas:


\begin{problema}
	Cambie el número de observaciones a 100,000.  Explique que es lo que ocurre en términos de:
	\begin{enumerate}
		\item El tiempo de ejecución para resolver el problema.
		\item El resultado final versus lo encontrado en clase:  es igual, o diferente...¿por qué?
		\item Las graficas  para representar los datos/resultados.
	\end{enumerate}
\end{problema}

\begin{problema}
	Cambie el número de observaciones a 1,000,000.  Explique que es lo que ocurre en términos de:
	\begin{enumerate}
		\item El tiempo de ejecución para resolver el problema.
		\item El resultado final versus lo encontrado en clase:  es igual, o diferente...¿por qué?
		\item Las graficas  para representar los datos/resultados.
	\end{enumerate}

\end{problema}

\begin{problema}
	“Juegue” un poco con el valor de la tasa de aprendizaje, por ejemplo  0.0001, 0.001, 0.1, 1.  Para cada uno de estos indique:
	\begin{enumerate}
		\item ¿Qué ocurre con el tiempo de ejecución?
		\item ¿Qué ocurre con la minimización de la pérdida?
		\item ¿Qué ocurre con los pesos y los sesgos?
		\item ¿Qué ocurre con las iteraciones?
		\item ¿El problema queda resuelto o no?  
		\item ¿Cuál es la apariencia de la última gráfica?  ¿Se cumple con la condición de que sea de 45 grados?
	\end{enumerate}
	
\end{problema}

\begin{problema}
	Cambie la función de pérdida “L2-norm” a la misma pero sin dividir por 2.  Explique lo que ocurre en términos de:
	\begin{enumerate}
		\item El tiempo que se tarda el algoritmo en terminar, comparado a lo que vimos en clase.
		\item Si la pérdida se minimiza igual que lo que vimos en clase.
		\item Si los pesos y sesgos son parecidos a los vistos en clase.
		\item Si el problema se resuelve como ocurrió en clase.
		\item Si se obtiene un mejor resultado al hacer más iteraciones.
	\end{enumerate}
\end{problema}

\begin{problema}
	Cambie la función de pérdida de la “L2-norm” a la “L1-norm”.  Explique lo que ocurre en términos de:
	
	\begin{enumerate}
		\item El tiempo que se tarda el algoritmo en terminar, comparado a lo que vimos en clase
		\item Si la pérdida se minimiza igual que lo que vimos en clase
		\item Si los pesos y sesgos son parecidos a los vistos en clase
		\item Si el problema se resuelve como ocurrió en clase
		 \item Si se obtiene un mejor resultado al hacer más iteraciones
		 \item  ¿Tendrá una de estas más limitaciones que la otra? 
	\end{enumerate}
\end{problema}

\begin{problema}
	Cree una función $f(x,z) = 13 * xs + 7 * zs  - 12$. 
	\begin{enumerate}
		\item ¿Funciona el algoritmo de la misma forma?
	\end{enumerate}
	
\end{problema}



%---------------------------

\end{document}